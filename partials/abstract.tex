% !TEX encoding = utf8
% !TEX root = ../main.tex

% This content has been generated automatically from https://www.docx2latex.com/docx2latex_free and https://github.com/MartinoMensio/doc2latex_process_chapters 
% Consider editing the source Google Doc instead of this one!



\section*{Context}
Conversational agents are now, more than ever, the answer to establish a seamless interaction between end users and any service out there. The spreading of these agents, also called \textit{bots} or \textit{chatbots}, has highlighted an important need: going beyond the simple (often pre-computed) answer and provide personalized answers according to users' profiles.

This work contains therefore two major themes - Natural Language Understanding (NLU) and personalization - that follow each other's in a traversal argumentation from the literature to the selected approaches.

The \textit{Natural Language Interface} enables a communication with the machine in the language that is really used by humans in their everyday interactions with other people. Traditional mobile applications force the human to imitate the computer to exchange information, performing precise questions in the form of commands. The goal of Conversational Interfaces instead, is to reverse this imitation process in order to bring the interaction closer to the users using his natural language.

Once this interaction channel is ready, with the mutual understanding of the involved parties, a process of \textit{personalization} can be applied in order to provide user-centric contents and interactions.

\section*{Goals}
The goals of this work are of two different natures: \textit{i)} analyze the State of the Art in order to identify the approaches that better suit the creation of a Conversational Agent \textit{ii)} build a bot prototype that uses the selected approaches.

The first goal includes the two main themes: Natural Language Understanding and personalization. For the first one, the focus stands on Recurrent Neural Network (RNN) approaches that can perform a sentence classification (intent detection task) and extraction of parameters (slot filling task). The literature also shows how the input words can be used, by exploiting features (word embeddings~\cite{pennington2014glove}) computed on large datasets of unlabeled text, to capture syntactic and semantic similarities on big dictionaries. These similarities allow to be more robust towards words that have not been considered in training sets. For the personalization theme instead, the Content-based and Collaborative filtering are explained along with available methods to extract user features. These user features can be of anagraphic type or can be more undeclared traits such as Big Five Personality Traits~\cite{goldberg1993structure} that can be computed by analyzing external data coming from social networks and interests.

The second goal, the running prototype, is focused on the only theme of Natural Language Understanding. As use case the domain of urban mobility is chosen, having as specific objective the design and implementation of a Conversational Agent to retrieve real-time information about the bike sharing system. The bot should offer capabilities to find bikes given the user location, or plan trips between different points of a given city, using the data from the bike sharing providers. To provide a natural interaction with a user, the bot should be able to process requests that are not in the strict form of commands, but in a more conversational fashion. To reduce the distance between the system and the user, the bot should support multilinguality, to increase the usability of the tool for audiences belonging to different nationalities; for this reason both English and Italian languages are targeted with the prototype.

\section*{Personal contributions and results achieved}
Moving on the fulfillment of these goals, the approach is structured as follows. Firstly the description focuses on Natural Language Understanding, describing the selected approach to target two types of interactions (single-turn where each sentence is processed independently and multi-turn where an interaction context exists): for the single-turn interactions the reference State of The Art joint approach~\cite{liu2016attention}\ is chosen, while for the multi-turn ones  it is proposed an extended version of the former.

Then, relatively to the bike sharing prototype, an analysis is done of the target scenarios where the bot is meant to be used. After giving a high-level view of the components required to interact with chat platforms (such as Facebook Messenger, Telegram, Skype) and the required services (bike sharing providers, directions provider), an overview of the designed intents and entities is given. For the personalization theme instead, two main needs are described: providing content recommendations, such as interesting places along the path of the cyclist, and tailoring the communication means to the needs of the user. The personalization theme actually is discussed with a sketch that can be used as guideline for implementation.

Moving to the implementation, a description is done of the details that made it possible to bring the bot prototype to life. The considered parts are the interactions with the chat platforms, the framework used for the NLU, and a dedicated section to word embeddings that shows how they have been computed on the Wikipedia Italian corpus\footnote{\url{https://dumps.wikimedia.org/}} to support this secondary language.

Finally, for the validation (both of the selected NLU approach and of the prototype) a description is done of the evaluation framework that has been used. For the evaluation of NLU, the datasets (standardized ones and collected ones for the specific bike sharing bot in both English and Italian) are explained and the measures are defined. The results underline the importance of features such as word embeddings or the structure of the network itself. For the overall evaluation of the prototype instead, it emerges how difficult it is to obtain human-like understanding performances.

 %%%%%%%%%%%%  Starting New Page here %%%%%%%%%%%%%%

